% !TEX TS-program = pdflatex
% !TEX encoding = UTF-8 Unicode

% This is a simple template for a LaTeX document using the "article" class.
% See "book", "report", "letter" for other types of document.

\documentclass[14pt]{article} % use larger type; default would be 10pt

\usepackage[utf8]{inputenc} % set input encoding (not needed with XeLaTeX)
\usepackage{listings}
\lstset{language=Java}

%%% Examples of Article customizations
% These packages are optional, depending whether you want the features they provide.
% See the LaTeX Companion or other references for full information.

%%% PAGE DIMENSIONS
\usepackage{geometry} % to change the page dimensions
\geometry{a4paper} % or letterpaper (US) or a5paper or....
% \geometry{margins=2in} % for example, change the margins to 2 inches all round
% \geometry{landscape} % set up the page for landscape
%   read geometry.pdf for detailed page layout information

\usepackage{graphicx} % support the \includegraphics command and options

% \usepackage[parfill]{parskip} % Activate to begin paragraphs with an empty line rather than an indent

%%% PACKAGES
\usepackage{booktabs} % for much better looking tables
\usepackage{array} % for better arrays (eg matrices) in maths
\usepackage{paralist} % very flexible & customisable lists (eg. enumerate/itemize, etc.)
\usepackage{verbatim} % adds environment for commenting out blocks of text & for better verbatim
\usepackage{subfig} % make it possible to include more than one captioned figure/table in a single float
% These packages are all incorporated in the memoir class to one degree or another...

%%% HEADERS & FOOTERS
\usepackage{fancyhdr} % This should be set AFTER setting up the page geometry
\pagestyle{fancy} % options: empty , plain , fancy
\renewcommand{\headrulewidth}{0pt} % customise the layout...
\lhead{}\chead{}\rhead{}
\lfoot{}\cfoot{\thepage}\rfoot{}

%%% SECTION TITLE APPEARANCE
\usepackage{sectsty}
\allsectionsfont{\sffamily\mdseries\upshape} % (See the fntguide.pdf for font help)
% (This matches ConTeXt defaults)

%%% ToC (table of contents) APPEARANCE
\usepackage[nottoc,notlof,notlot]{tocbibind} % Put the bibliography in the ToC
\usepackage[titles,subfigure]{tocloft} % Alter the style of the Table of Contents
\renewcommand{\cftsecfont}{\rmfamily\mdseries\upshape}
\renewcommand{\cftsecpagefont}{\rmfamily\mdseries\upshape} % No bold!

%%% END Article customizations

%%% The "real" document content comes below...

\title{Union Find: Connectivity}
\author{B. Smith}
%\date{} % Activate to display a given date or no date (if empty),
         % otherwise the current date is printed 

\begin{document}
\maketitle

\section{Motivation}

\begin{itemize}
\item Why do we study algorithms?  
\item 

\end{itemize}

\subsection{The meaning of efficiency: the n-body problem of physics }


\begin{center}
$F_1 = F_2 = G \frac{m_1 m_2}{r^2}$
\end{center}



\begin{verbatim}
\end{verbatim}



\subsection{Closed Form for Arithmetic Series}


\subsection{How much time is it really?}



\subsection{Growth of Functions: Profiles of Functions}

\begin{lstlisting} [frame=single]
int linSearch(int target)
{
  //return the position of the found item
  //otherwise return -1 if not found

  int pos = -1;

  for(int i =0; i< list.length; i++)
  {
    if( list[i] == target)
    {  pos = i;
       break;
    }
  }
  return pos;
}
\end{lstlisting}




\begin{lstlisting} [frame=single]
int binSearch( int target)
{
  while ( r >= l )
  {
    int m = (l+r)/2;
    if ( list[m] == target) return m;
    if ( list[m] > target )  l = m + 1;
       else  r = m - 1;
  }
return -1;
}
 \end{lstlisting}
 

 
0  1  2   3   4    5   6    7     (index)  \\
3  6  7   9  15  17  18   21   (array contents)

l = 0, r = 7:    m = (0+7)/3,  list[m] = 9  \\
- since \lstinline!  target < 9!,   ( ie $6 < 9$)  \\  
- look in left half and repeat steps  \\ 


\lstinline! r=m-1! yields  3-1 =2  \\
\lstinline! l=0, r=2!:       m = (0+2)/2 =1,  list[m] = 6  \\





\end{document}
